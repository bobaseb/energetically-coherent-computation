\section{Consciousness as Non-computational}

ECC's proposal represents a distinctive philosophical approach to consciousness that diverges from traditional computationalist views while maintaining a firmly physicalist stance \cite{piccinini2020neurocognitive}. At its core, ECC posits that consciousness emerges not from abstract information processing or symbolic manipulation, but from coherent energy flows within biological systems. This philosophical framework challenges both classical functionalism and computational theories of mind by emphasizing the irreducible role of physical embodiment and energetic dynamics in conscious experience \cite{thompson2001radical}.

ECC's philosophical commitments can be understood through three fundamental principles. First, consciousness requires specific forms of energetic coherence that cannot be reduced to computational processes alone \cite{van1995might}. Unlike traditional functionalist approaches that treat consciousness as substrate-independent, ECC argues that conscious experience is inherently tied to particular physical and energetic configurations, typically found in biological systems. Second, ECC maintains that consciousness operates through a rich alphabet of energetic states, shaped by transcriptomic profiles and molecular diversity, rather than through binary or discrete symbolic representations \cite{wheeler2005reconstructing}. Third, consciousness emerges as a field-like phenomenon characterized by continuous, dynamic coherence rather than discrete state transitions.

This philosophical stance positions ECC as a unique bridge between physicalist and emergentist views of consciousness \cite{jonas2001phenomenon}. While firmly grounded in physical processes, ECC suggests that conscious experience emerges from specific organizations of energy flows. Where the dynamics of said energy flows may admit a computational description, conscious experience itself cannot be captured by purely computational or mechanistic descriptions. This approach offers a novel solution to classical philosophical problems such as the symbol grounding problem \cite{harnad1990symbol} and the hard problem of consciousness \cite{chalmers1997conscious}, by rooting conscious experience in concrete, physically realized energy dynamics rather than abstract computational processes.

This physicalist yet non-computationalist approach has significant implications for several longstanding debates in philosophy of mind. First, regarding multiple realizability—a cornerstone of traditional functionalism—ECC takes a more constrained position \cite{piccinini2013neural, anderson2024physical}. While conscious states might be realizable in different physical substrates, ECC argues that these substrates must be capable of supporting specific types of energetic coherence and dynamic stability. This suggests that consciousness cannot be implemented in just any computational system, but requires materials and organizations capable of sustaining coherent energy flows similar to those found in biological brains \cite{van1995might}.

ECC's stance on the mind-body problem is particularly distinctive. Rather than treating consciousness as an emergent property of computational processes \cite{piccinini2020neurocognitive} or as a fundamental feature of all matter \cite{Goff2019}, ECC suggests that consciousness arises specifically from coherent energy dynamics within systems that maintain low-entropy, stable states. This view acknowledges the physical basis of consciousness while recognizing that not all physical systems—even those capable of complex information processing \cite{tononi2016integrated}—will necessarily give rise to conscious experience. The key distinction lies in a system's ability to achieve and maintain energetic coherence across multiple scales \cite{horst2011symbols}.

The framework also offers fresh insights into the nature of qualia or phenomenal experience. Instead of treating qualia as computational states or abstract representations \cite{bishop2009computers}, ECC grounds them in specific patterns of energetic coherence shaped by transcriptomic profiles and molecular diversity. This approach suggests that the qualitative aspects of conscious experience are neither mysterious nor epiphenomenal, but are direct manifestations of structured energy flows within biological systems \cite{noe2009out}. Such a view helps bridge the explanatory gap between physical processes and phenomenal experience by identifying consciousness with particular forms of energetic organization.

Perhaps most significantly, ECC's philosophical framework challenges us to rethink the relationship between function and implementation in conscious systems \cite{piccinini2013neural}. While traditional approaches have often treated these as separable—with function being abstractable from physical implementation—ECC suggests they are fundamentally intertwined when it comes to consciousness. The specific energetic dynamics that give rise to conscious experience cannot be separated from their physical realization without losing essential properties that make consciousness possible \cite{van1995might}. This represents a form of embodied functionalism that recognizes the inseparability of conscious functions from their physical instantiation.

This philosophical stance has profound implications for artificial consciousness and cognitive science. It suggests that creating conscious artificial systems would require not just implementing the right algorithms or information processing architecture \cite{searle1980minds} (cf. \cite{butlin2023consciousnessartificialintelligenceinsights}), but engineering physical systems capable of sustaining the specific types of energetic coherence found in biological brains. This moves beyond the traditional artificial intelligence paradigm of abstract computation toward a more biologically-inspired approach that emphasizes physical dynamics and energy flows \cite{dreyfus1992computers}.

ECC thus presents a philosophical framework that is at once physicalist and non-reductionist, acknowledging both the material basis of consciousness and the impossibility of reducing it to purely computational descriptions \cite{horst2011symbols}. It offers a middle path between eliminative materialism \cite{churchland1986neurophilosophy} and dualism \cite{chalmers1997conscious}, suggesting that consciousness is neither illusion nor magic, but rather a physical phenomenon requiring specific forms of energetic organization and coherence \cite{jonas2001phenomenon}.

The classical computational framework, articulated through universal machines \cite{turing1936computable} and formalized cognitive processes (e.g., \cite{marr1982vision}), has dominated functionalist accounts of mind \cite{putnam1988representation}. This computational paradigm suggests that any cognitive process can be understood as an algorithm operating on representations. As mentioned above, this framework faces fundamental challenges when applied to consciousness \cite{fodor2000mind}.

Not every natural process requires or admits computational description \cite{rosen1991life}. Just as digestion cannot be adequately characterized as information processing, and gravitational phenomena cannot be reduced to computation, conscious experience may emerge from physical dynamics that resist computational abstraction. This aligns with later skepticism about the computational theory of mind, particularly regarding the context-sensitivity and holistic nature of conscious thought \cite{gibson2014ecological}. The framework suggests that attempting to reduce consciousness to computation represents a category error - confusing the abstract map of computational description with the physical territory of conscious experience. Furthemore, though we accept that the energy flows that support consciousness may admit a computational description, they are not equivalent to them.

This perspective helps resolve certain paradoxes in functionalist theories of mind \cite{piccinini2020neurocognitive}. Rather than treating consciousness as substrate-independent computation, ECC suggests it emerges from specific patterns of energetic coherence that remain grounded in physical dynamics. This maintains functionalism's key insight about the importance of organization while avoiding what has been identified as "computational chauvinism" - the tendency to treat all cognitive processes as fundamentally computational \cite{bishop2009computers}. The framework indicates that while some mental processes may be computational in nature, consciousness itself requires physical dynamics that exceed purely computational description.

Traditional functionalism has become so intertwined with computational theory of mind that the two are often treated as inseparable \cite{piccinini2020neurocognitive}. However, ECC suggests a novel approach: a functionalism grounded in energetic coherence rather than abstract computation. This reformulation maintains functionalism's core insight—that mental states are defined by their functional roles—while departing from the assumption that these roles must be realized through computational processes \cite{van1995might}.

In this reconceptualization, mental functions are understood not as algorithmic operations (in an ontological sense, though they can be in an epistemological one) but as patterns of coherent energy flows within physically structured systems \cite{thompson2001radical}. Where traditional functionalism might describe perception as information processing, ECC might characterize it as the achievement and maintenance of specific energetic configurations that faithfully represent environmental stimuli \cite{gibson2014ecological}. Similarly, memory becomes not the storage and retrieval of symbolic information, but the stabilization and reactivation of particular energetic patterns within the brain's coherent field.

This \textit{energetic functionalism} differs crucially from computational functionalism in its treatment of implementation \cite{horst2011symbols}. While computational functionalism suggests that any substrate capable of implementing the right algorithms could support consciousness, energetic functionalism argues that conscious functions require specific physical conditions that enable coherent energy flows. The function cannot be abstracted from its physical realization because the very nature of the function—the maintenance of coherent, low-entropy states—depends on particular physical and energetic properties \cite{rosen1991life}.

This energetic functionalism differs crucially from computational functionalism in its treatment of implementation. While computational functionalism suggests that any substrate capable of implementing the right algorithms could support consciousness \cite{wheeler2010defense}, energetic functionalism argues that conscious functions require specific physical conditions that enable coherent energy flows. The function cannot be abstracted from its physical realization because the very nature of the function—the maintenance of coherent, low-entropy states—depends on particular physical and energetic properties \cite{nicholson2018everything,whitehead2010process}.

To recap, this reformulation addresses several longstanding challenges in functionalist theory \cite{polger2016multiple}. First, it offers a solution to the symbolic grounding problem by rooting mental functions in physically realized energy dynamics rather than abstract symbols. In ECC's energetic functionalism, meaning and representation are not arbitrary mappings requiring external grounding, but emerge directly from the brain's coherent energy states shaped by transcriptomic profiles and molecular diversity \cite{gillett2016reduction}. The rich alphabet of possible states provides an intrinsically grounded basis for representation without requiring computational abstraction.

Moreover, energetic functionalism provides new insights into the unity of consciousness—a phenomenon that has proven difficult to explain within traditional computational frameworks \cite{van1998dynamical}. Rather than requiring a central processor or global workspace \cite{Baars2013} to integrate discrete computational processes, ECC suggests that unity emerges naturally from the continuous, field-like properties of coherent energy flows \cite{McFadden2020}. The brain's capacity to maintain coherent states across multiple scales creates a unified conscious field without needing additional mechanisms to bind separate processes together \cite{thompson2011living}.

This approach also offers a more nuanced view of multiple realizability. While maintaining that conscious functions could potentially be realized in different physical substrates, energetic functionalism argues that these substrates must be capable of supporting specific types of energetic coherence. This suggests a constrained form of multiple realizability where conscious functions are replicable only in systems that can achieve and maintain the necessary patterns of energy flow. Such systems might include both biological brains and specially engineered artificial structures, but would exclude traditional digital computers that operate through discrete state transitions \cite{wheeler2010defense}.

Energetic functionalism also provides new perspectives on the relationship between consciousness and physical implementation \cite{mossio2015biological}. Unlike computational functionalism, which treats implementation details as largely irrelevant to mental functions, ECC suggests that certain physical properties—particularly those that enable coherent energy flows—are essential to conscious functions. This doesn't reduce mental states to physical states in a simple type-identity fashion, but rather suggests that conscious functions require specific classes of physical organization that support energetic coherence \cite{dupre2012processes}.

This view has important implications for artificial consciousness. Rather than focusing on replicating computational algorithms, the development of conscious artificial systems would require engineering physical substrates capable of supporting coherent energy dynamics similar to those found in biological brains \cite{chemero2013radical}. This might involve creating new kinds of materials and architectures that can maintain low-entropy, coherent states across multiple scales. The goal would not be to simulate consciousness computationally, but to create physical systems that can achieve and maintain the kinds of energetic coherence necessary for conscious experience \cite{hutto2012radicalizing}.

The shift from computational to energetic functionalism suggests a novel approach to understanding the nature of consciousness and its relationship to physical systems \cite{nicholson2018everything}. This reconceptualization naturally leads us to consider how this view relates to traditional debates about type and token identity theories in the philosophy of mind. While traditional type identity theory suggests a one-to-one correspondence between mental and physical states, and token identity theory allows for multiple physical realizations of the same mental state, ECC's approach suggests a more nuanced view based on classes of energetic coherence \cite{gillett2016reduction}.