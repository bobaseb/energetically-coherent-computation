\section{Mathematical Details}

1. Sheaf-Theoretic Foundations

Let $(X,\tau)$ be the topological space representing the cortical sheet. For each open set $U \in \tau$, we define a sheaf $\mathcal{F}$ that assigns:

$\mathcal{F}(U) = \{s: U \rightarrow E \mid s \text{ is a section over } U\}$

where $E$ represents the fiber bundle of possible energy states. The key sheaf axioms must be satisfied:

(S1) Locality: For any open cover ${U_i}$ of $U$ and sections $s,t \in \mathcal{F}(U)$, if $s|U_i = t|U_i$ for all $i$, then $s = t$.

(S2) Gluing: Given sections $s_i \in \mathcal{F}(U_i)$ that agree on overlaps $U_i \cap U_j$, there exists a unique section $s \in \mathcal{F}(U)$ such that $s|U_i = s_i$.

For conscious states, we require additional coherence conditions:

$\|s|_{U \cap V} - t|_{U \cap V}\| \leq \varepsilon(\mu(U,V))$

\text{where:}
\begin{itemize}
\item $\mu(U,V)$ measures the ``distance" between regions
\item $\varepsilon(\cdot)$ is a monotonically increasing function
\item $\|\cdot\|$ is an appropriate norm on the space of sections
\end{itemize}

2. Stress-Energy Tensor Framework

The full stress-energy tensor takes the form:

$T_{\mu\nu} = \sum_i \alpha_i(x)T^{(i)}_{\mu\nu} + \sum_{i,j} \beta_{ij}(x)C^{(i,j)}_{\mu\nu}$

\text{where:}
\begin{itemize}
\item $T^{(i)}_{\mu\nu}$ represents individual subsystem contributions
\item $C^{(i,j)}_{\mu\nu}$ captures coupling terms
\item $\alpha_i(x)$, $\beta_{ij}(x)$ are position-dependent weighting functions
\end{itemize}

3. Triangulation and Recursive Dynamics

The triangulation operator $T$ acting on three regions $A$,$B$,$C$ must satisfy:

(T1) Symmetry: $T(A,B,C) = T(B,C,A) = T(C,A,B)$

(T2) Distance-dependent coherence:
$\|T(A,B,C) - T(A,B',C)\| \leq K \cdot \exp(-\lambda d(B,B'))$

where $K,\lambda > 0$ are constants and $d(\cdot,\cdot)$ is an appropriate metric.

The recursive update operator $R$ satisfies:

$R^{(n+1)}(x) = F[\{R^{(n)}(y)\}_{y \in N(x)}]$

with convergence condition:
$\lim_{n \rightarrow \infty} \|R^{(n+1)} - R^{(n)}\| = 0$

4. Interface and Coupling Terms

For interfaces between regions $\Omega_1$,$\Omega_2$, the boundary terms take the form:

$B_{\mu\nu} = \sigma(x)[n \cdot \nabla T_{\mu\nu}] + \kappa(x)T_{\mu\nu}|_{\partial\Omega}$

subject to the conservation law:

$\int_{\partial\Omega} B_{\mu\nu} \,dS + \int_\Omega (\partial_\lambda T_{\mu\nu})^\lambda \,dV = 0$

The coupling terms between subsystems $i$,$j$ satisfy:

$C_{\mu\nu}(i,j) = \gamma_{ij}[\partial_\mu\phi_i\partial_\nu\phi_j - \tfrac{1}{2}g_{\mu\nu}(\partial_\lambda\phi_i\partial^\lambda\phi_j)]$

\text{where:}
\begin{itemize}
\item $\phi_i,\phi_j$ are field variables
\item $\gamma_{ij}$ is the coupling strength
\item $g_{\mu\nu}$ is the metric tensor
\end{itemize}

5. Global Coherence Conditions

The global section $S$ must satisfy simultaneous constraints:

(C1) Local-to-global compatibility:
$\text{For all } U \subseteq X: \|S|_U - s_U\| \leq \eta(U)$

(C2) Energy conservation:
$\nabla_\mu T^{\mu\nu} + \sum_{i,j} C_{\mu\nu}(i,j) = 0$

(C3) Thermodynamic bounds:
$-\int_X T_{\mu\nu}\nabla^\mu\xi^\nu \,dV \leq \eta_{\text{max}}$

6. Stability Analysis

For perturbations $\delta s$ around equilibrium states:

(S1) Linear stability:
$\|R^{(n)}(s + \delta s) - R^{(n)}(s)\| \leq \exp(-\lambda n)\|\delta s\|$

(S2) Local-global stability:
$\partial_t\|S - \sum_i s_i\phi_i\| \leq -\alpha\|S - \sum_i s_i\phi_i\| + \beta$

\text{where:}
\begin{itemize}
\item $\phi_i$ are local basis functions
\item $\alpha,\beta > 0$ are stability parameters
\item $s_i$ are local sections
\end{itemize}

7. Convergence Theorems

Theorem 1 (Global Existence): Under conditions (C1)-(C3), there exists a unique global section S satisfying all coherence constraints.

Theorem 2 (Stability): Solutions satisfying (S1)-(S2) converge to stable conscious states exponentially fast in time.

Note: Full proofs of these theorems require additional technical machinery beyond the scope of this appendix. For detailed proofs and further mathematical development, see referenced technical papers.

Several foundational works provide essential background for understanding the mathematical framework employed in ECC. The classical treatment in \cite{bredon1997sheaf} provides comprehensive coverage of sheaf theory and its applications to topology, particularly valuable for understanding the local-to-global aspects of conscious integration. For stress-energy tensors and their applications in physical systems, \cite{misner2017gravitation} offers fundamental insights into how energy flows can be mathematically characterized. The treatment of categories and sheaves in \cite{kashiwara2006categories} provides crucial mathematical tools for understanding how local conscious states combine into global experience. \cite{marsden2013introduction} develops the essential framework for understanding symmetry and conservation principles in dynamical systems, particularly relevant for consciousness as an energetically coherent phenomenon. For those interested in the geometric aspects, \cite{wells2008differential} provides crucial background on differential geometry and its applications to complex systems. The relationship between physics and computation is thoughtfully explored in \cite{baez2011physics}, offering important perspectives on how mathematical structures can bridge physical and computational descriptions of consciousness. These works collectively establish the mathematical foundations necessary for understanding how consciousness emerges from coherent energy dynamics in biological systems.