\section{Technical Specifications}

I. Energetic Parameters

A. Temporal Scale

- Ultra-fast: 0.1-1ms (action potentials)

- Fast: 1-10ms (synaptic transmission)

- Intermediate: 10-100ms (local field potentials)

- Slow: 100ms-1s (astrocytic calcium waves)

- Very slow: 1-10s (metabolic fluctuations)

B. Spatial Scales

- Molecular: 1-10nm (protein conformations)

- Subcellular: 100nm-1$\mu$m (synaptic structures)

- Cellular: 1-100$\mu$m (neuronal/glial bodies)

- Local circuits: 100$\mu$m-1mm (cortical columns)

- Regional: 1-10mm (cortical areas)

- Global: >10mm (whole-brain dynamics)

II. Biophysical Constraints

A. Thermodynamic Parameters

- Operating temperature range: 35-39°C

- Local temperature gradients: ±0.5°C

- Maximum sustainable entropy production rate

- Critical coherence thresholds

- Metabolic energy density limits

B. Field Properties

- Electric field strength ranges

- Magnetic field intensity bounds

- Chemical gradient maxima

- Mechanical stress limits

- Interface coupling strengths

- Field penetration depths

III. Implementation Requirements

Cellular Architecture

The maintenance of conscious states requires specific cellular density distributions. Neuronal density must exceed 20,000 cells per cubic millimeter in cortical regions supporting conscious processing. Astrocyte-to-neuron ratios should maintain approximately 1.4:1 proportion to support adequate metabolic and ionic homeostasis. Gap junction density between astrocytes must achieve minimum coupling strength for calcium wave propagation.

Membrane Properties

Cellular membranes must maintain resting potential between -60mV and -70mV with maximum fluctuation tolerance of ±5mV. Receptor density requires minimum 1000 functional proteins per square micron for primary neurotransmitter systems. Local membrane capacitance should remain within 0.8-1.0 $\mu$F/cm² for proper signal propagation.

Metabolic Parameters

Oxygen consumption must sustain 3-4 $\mu$mol per gram of tissue per minute under baseline conditions with capacity to increase threefold during peak activation. Glucose utilization requires maintenance of 5-6 $\mu$mol per gram per minute with local reserves sufficient for 30 seconds of maximal activity. Lactate shuttle systems must support clearance rates matching peak neuronal metabolic demand.

Signal Propagation

Action potential conduction velocities must maintain 1-100 meters per second depending on fiber type and diameter. Local field potential propagation requires phase velocities between 0.1-1 meters per second for maintaining coherence. Calcium wave propagation through astrocytic networks should achieve 10-20 micrometers per second for proper metabolic coordination.

IV. System Integration Parameters

Coherence Maintenance

Regional coherence requires minimum phase synchronization of 0.7 (measured by phase-locking value) across functionally connected areas. Cross-frequency coupling must maintain stable phase-amplitude relationships within defined frequency bands: theta (4-8 Hz) to gamma (30-100 Hz) coupling requires modulation index above 0.3. Maximum allowable phase lag between reciprocally connected regions cannot exceed 20 milliseconds.

Interface Dynamics

Synaptic interfaces must maintain vesicle release probability between 0.2-0.8 depending on synapse type. Neurotransmitter clearance rates should achieve 80\% removal within 500 microseconds for fast synapses. Astrocytic endfeet must cover minimum 60\% of synaptic interfaces for proper neurotransmitter and ion regulation.

Field Integration

Electromagnetic field integration requires minimum spatial coherence length of 0.5 millimeters under normal operating conditions. Field strength gradients must not exceed 10\% change per 100 micrometers to maintain stable coupling. Temporal field stability requires autocorrelation time exceeding 50 milliseconds for conscious integration.

Homeostatic Regulation

Ion concentration gradients must maintain sodium/potassium ratios within 5\% of optimal values. pH regulation requires buffering capacity sufficient to maintain values between 7.0-7.4 with maximum deviation of ±0.1 units. Calcium wave refractory periods must exceed 15 seconds to prevent reverberatory activation.

V. System Boundaries and Limitations

Operational Constraints

Maximum sustainable conscious processing cannot exceed 40 Hz for integrated experiences. Information integration capacity limits approximate 7±2 distinct elements within single conscious frames. Temporal binding window for unified conscious experience spans 100-150 milliseconds. System can tolerate local disruptions up to 30\% of normal activity while maintaining global coherence.

Physical Boundaries

Neural light cone propagation limited by axonal conduction velocities and synaptic delays. Conscious integration cannot exceed speed of slowest necessary component (typically calcium wave propagation). Maximum spatial separation for direct causal influence approximately 4-5 millimeters without intermediate processing stages.

Resource Requirements

Continuous consciousness requires minimum 20\% of total body glucose consumption. Oxygen delivery must maintain partial pressure above 25 mmHg in neural tissue. ATP production must sustain 10\^23 molecules per gram tissue per minute. Local energy stores must buffer supply/demand mismatches for minimum 30 seconds.

Failure Modes

System coherence breaks down when thermal fluctuations exceed 1°C from baseline. Critical failure occurs if ATP levels fall below 50\% normal for more than 10 seconds. Information integration collapses with >40\% disruption of normal connectivity patterns. Consciousness cannot be maintained if more than 25\% of regional interfaces lose coherence simultaneously.

These specifications represent minimum requirements for conscious processing. Actual biological systems typically maintain significant safety margins above these thresholds during normal operation.

Suggested Reading

Several foundational works provide crucial understanding of the technical parameters and constraints governing conscious systems. For cellular biophysics and neural energetics, \cite{attwell2001energy} offers comprehensive analysis of energy budgets in neural tissue, while \cite{magistretti2015cellular} provides essential insights into brain energy metabolism and its relationship to neural function. The dynamics of neural systems are thoroughly explored in \cite{buzsaki2006rhythms}, offering crucial perspectives on how different frequency bands contribute to conscious processing. \cite{harris2012synaptic} provides detailed analysis of synaptic energy use and supply, essential for understanding the energetic constraints on neural computation. For understanding the relationship between metabolism and brain function, \cite{herculano2011scaling} offers important insights into how energy budgets scale across different brain organizations. The biophysical features of neural systems are thoughtfully explored in \cite{georgiou2012biophysical}, providing crucial details about the physical constraints governing neural computation. These works collectively establish the technical foundations necessary for understanding the physical and biological requirements for maintaining conscious states through energetically coherent processing.