\section{Stress-energy Tensor}

The stress-energy tensor $T\mu\nu$ describes the flow and distribution of energy-momentum within conscious systems. For our purposes, we decompose it into subsystem-specific components:

$T_{\mu\nu} = T_{\mu\nu}^{(EM)} + T_{\mu\nu}^{(chem)} + T_{\mu\nu}^{(mech)} + T_{\mu\nu}^{(int)}$

where the superscripts denote electromagnetic, chemical, mechanical, and interaction terms respectively. The Jacobian of this tensor, $\partial_\sigma T_{\mu\nu}$, captures how energy flows change across space and time:

$\partial_\sigma T_{\mu\nu} = \frac{\partial T_{\mu\nu}}{\partial x^\sigma}$

This framework allows us to track energy flow patterns essential for conscious processing while maintaining compatibility with the sheaf-theoretic structure previously described.

\subsection{Jacobian of the Stress-Energy Tensor}

The Jacobian of the stress-energy tensor provides a mathematical framework for tracking how conscious processing emerges from coordinated energy flows. For each point $x$ in the brain's neural architecture, we can express the full dynamics through:

$\partial_\sigma T_{\mu\nu}(x) = \begin{pmatrix} 
\frac{\partial T_{00}}{\partial t} & \frac{\partial T_{0i}}{\partial t} & \frac{\partial T_{0j}}{\partial t} \\[1em]
\frac{\partial T_{i0}}{\partial x^k} & \frac{\partial T_{ij}}{\partial x^k} & \frac{\partial T_{ij}}{\partial x^l}
\end{pmatrix}$

where each component describes specific aspects of conscious computation:

1. Energy Flow Components:

- $\frac{\partial T_{00}}{\partial t}$ tracks changes in local energy density

- $\frac{\partial T_{0i}}{\partial x^k}$ describes spatial energy flux gradients

- $\frac{\partial T_{0i}}{\partial x^k}$ captures stress propagation

2. Subsystem-Specific Terms:
For each subsystem $\alpha$ (electromagnetic, chemical, mechanical), we have:

$\partial_\sigma T_{\mu\nu}(\alpha) + \sum_\beta C_{\mu\nu}(\alpha,\beta) = J_{\mu\nu}(\alpha)$

\text{where:}
\begin{itemize}
\item $C_{\mu\nu}(\alpha,\beta)$ represents coupling terms between subsystems
\item $J_{\mu\nu}(\alpha)$ represents the source terms
\end{itemize}

3. Interface Terms:
The framework introduces boundary terms $B_{\mu\nu}$ that capture energy exchange at interfaces between regions:

$B_{\mu\nu}(x) = [T_{\mu\nu}(\alpha)]_{\partial\Omega}$

These interface terms are crucial for maintaining coherent conscious processing across regional boundaries.

\begin{figure}[h]
    \centering
    \includegraphics[width=0.8\textwidth]{jacobian.png}

    \caption{The Jacobian of the stress-energy tensor specifies the necessary information to describe energy flows in the brain.}
\end{figure}

\subsection{Coupling-interface Terms}

The coupling and interface terms in the stress-energy framework capture how different subsystems interact to maintain conscious coherence. These terms are essential for understanding how local energy dynamics combine into global conscious states.

For any two subsystems $\alpha$ and $\beta$, the coupling terms $C_{\mu\nu}(\alpha,\beta)$ can be expanded as:

$C_{\mu\nu}(\alpha,\beta) = \gamma(\alpha,\beta)[\partial_\nu\phi(\alpha)\partial_\mu\phi(\beta) - \eta_{\mu\nu}(\partial_\lambda\phi(\alpha)\partial^\lambda\phi(\beta))]$

\text{where:}
\begin{itemize}
\item $\gamma(\alpha,\beta)$ represents coupling strength
\item $\phi(\alpha)$, $\phi(\beta)$ are field variables for each subsystem
\item $\eta_{\mu\nu}$ is the metric tensor
\item $\partial_\lambda$ denotes covariant derivatives
\end{itemize}

The interface terms at boundaries $\partial \Omega$ take the form:

$B_{\mu\nu}(x) = \sigma(x)[n \cdot \nabla T_{\mu\nu}] + \kappa(x)T_{\mu\nu}|_{\partial\Omega}$

\text{where:}
\begin{itemize}
\item $\sigma(x)$ represents interface conductivity
\item $n$ is the unit normal to the boundary
\item $\kappa(x)$ captures boundary resistance
\end{itemize}

These terms must satisfy conservation conditions:

$\int_{\partial\Omega} B_{\mu\nu}(x)\,dS + \int_\Omega C_{\mu\nu}(\alpha,\beta)\,dV = 0$

This ensures energy and information flow continuously across subsystem boundaries while maintaining coherent conscious states.

The total dynamics at interfaces must also satisfy coherent boundary conditions:

$\sum_{\alpha,\beta} [C_{\mu\nu}(\alpha,\beta) + \partial_\lambda B_{\mu\nu}(\alpha,\beta)] \leq \eta(x,t)$

where $\eta (x,t)$ represents the maximum allowable local deviation from perfect coherence, constrained by thermodynamic considerations. These conditions ensure that energy transfers between subsystems remain within bounds that support conscious processing.

The complete interface dynamics can be expressed through a hierarchy of coupling terms:

$C_{\mu\nu} = C^{(1)}_{\mu\nu} + C^{(2)}_{\mu\nu} + C^{(3)}_{\mu\nu} + \cdots$

where each order captures increasingly complex interactions between subsystems, with higher-order terms typically decreasing in magnitude:

$|C^{(n)}_{\mu\nu}| \sim O(\gamma^n)$

where $\gamma < 1$ is a coupling parameter.

The mathematical foundations of stress-energy tensors and their applications to complex dynamical systems can be further explored through several seminal works. A comprehensive introduction to the geometric foundations is presented in \cite{Frankel2011}, which provides essential background on tensor analysis and differential geometry. The classic treatment in \cite{Misner1973} offers deep insights into how stress-energy tensors describe the flow and distribution of energy-momentum in physical systems, while \cite{Landau1987} provides crucial perspectives on how these concepts apply to continuous media and field theories. For those interested in the quantum aspects of energy flow and coherence, \cite{Peskin1995} develops the mathematical framework necessary for understanding field-theoretic approaches to information integration. The relationship between stress-energy tensors and general covariance principles is thoroughly examined in \cite{Wald1984}, providing important context for understanding how local energy conservation emerges from global symmetries. These texts collectively establish the mathematical rigor needed to apply stress-energy tensor analysis to complex biological systems like the brain, where multiple forms of energy flow must maintain coherent relationships across various scales.