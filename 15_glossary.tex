\h{Glossary}

Animatism: A theoretical perspective that recognizes impersonal force or power in nature, reinterpreted in ECC as the recognition of coherent energy patterns in environmental systems.

Astrocytic Syncytia: Networks of interconnected astrocyte cells joined by gap junctions that enable direct cytoplasmic continuity and coherent energy distribution across neural tissue.

Basal Cognition: Fundamental information processing capabilities present at the cellular level, independent of neurons or synapses, as described in the work of Michael Levin.

Biological Naturalism: Philosophical position that consciousness is irreducibly grounded in biological processes, aligned with ECC's emphasis on physical implementation while extending beyond neural systems.

Bricoleur: In Lévi-Strauss's terminology, reinterpreted through ECC as a mode of thought maintaining closer connection to physically indexed states while creating new configurations.

Chemputation: The encoding and processing of information through molecular interactions and chemical states, particularly in cellular systems and biological computation.

Coherence Gradient: The systematic variation in energetic coherence across neural tissue that enables both local specialization and global integration of conscious states.

Conscious Bottleneck: The apparent limitation in conscious processing capacity, explained in ECC as arising from constraints on maintaining coherent energy states.

Dimensional Reduction: The process by which complex, high-dimensional patterns of neural activity are transformed into lower-dimensional conscious experiences while maintaining essential information.

Dissipative Coherence: The maintenance of stable conscious states through controlled energy dissipation, enabling both stability and adaptability.

Energetic Coherence: The stable organization of energy flows within biological systems that enables conscious processing, characterized by low-entropy states maintained through continuous feedback.

Ephaptic Coupling: Direct influence between neurons through local electric fields, independent of synaptic transmission, supporting field-like properties of consciousness.

Extracellular Matrix: The complex network of proteins and molecules outside cells that shapes energy flow patterns and influences conscious processing.

Field Integration: The process by which different forms of energy fields (electromagnetic, chemical, mechanical) combine to support conscious states.

Free Variable: In ECC, mental representations that can be manipulated independently of immediate physical grounding, enabling abstract thought while requiring active maintenance.

Gap Junction: Specialized protein channels between cells that allow direct electrical coupling and molecular exchange, crucial for maintaining coherent energy states.

Local-to-Global Coherence: The mechanism by which localized patterns of energetic coherence combine to create unified conscious experiences while maintaining distinct properties.

Mutual Recursion: The continuous feedback between different brain regions that enables stable conscious states through reciprocal influence and adjustment.

Neural Light Cone: The spatiotemporal boundary within which conscious integration can occur, determined by the speed of neural signal propagation and the maintenance of energetic coherence.

Neuropil: The dense network of neuronal and glial processes, synaptic connections, and extracellular matrix that provides the physical substrate for conscious processing.

Participation: Lévy-Bruhl's concept reinterpreted through ECC as representing specific patterns of coherence that integrate experience differently from analytical thought.

Physically Indexed State: Mental representations directly grounded in patterns of energetic coherence arising from cellular processes and neural architecture.

Recursion Stability: The property of conscious systems to maintain coherent states through repeated cycles of mutual influence and adjustment.

Rich Alphabet: The diverse repertoire of possible conscious states enabled by region-specific transcriptomic profiles and molecular diversity in neural tissue.

Sheaf Theory: Mathematical framework used in ECC to describe how local patterns of coherence combine into global conscious states while maintaining consistency across boundaries.

Stress-Energy Tensor: Mathematical tool used to analyze the flow and distribution of energy-momentum within conscious systems, incorporating electromagnetic, chemical, and mechanical components.

Thermal Noise Threshold: The level of thermal fluctuation that defines boundaries between conscious and unconscious processing in neural systems.

Transcriptomic Profile: The specific pattern of gene expression in neural tissue that shapes its capacity for information processing and energy management.

Triangulation: In ECC, the process by which different brain regions maintain consistent patterns of coherence through continuous mutual feedback and adjustment.

Wetware Computing: Hybrid computational systems that combine biological components with engineered structures to achieve specific forms of energetic coherence.